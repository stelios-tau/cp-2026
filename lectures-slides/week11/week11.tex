
\PassOptionsToPackage{colorlinks,linkcolor={blue},citecolor={blue},urlcolor={blue},breaklinks=true,final}{hyperref}
\PassOptionsToPackage{dvipsnames}{xcolor}
\documentclass[xcolor={dvipsnames,svgnames},aspectratio=169]{beamer}

\usepackage{fontawesome5}
\usepackage{booktabs} % For better table formatting
\usepackage{listings}

\lstset{
  tabsize = 4, %% set tab space width
  showstringspaces = false, %% prevent space marking in strings, string is defined as the text that is generally printed directly to the console
  numbers = left, %% display line numbers on the left
  commentstyle = \color{purple!60}, %% set comment color
  keywordstyle = \color{blue}, %% set keyword color
  stringstyle = \color{red}, %% set string color
  rulecolor = \color{black}, %% set frame color to avoid being affected by text color
  basicstyle = \small \ttfamily , %% set listing font and size
  breaklines = true, %% enable line breaking
  numberstyle = \tiny,
}

\title{Concurrent Programming}
\subtitle{Week 11 (Lecture 4)}
\author{Stelios Tsampas}
\institute{
  \faEnvelope \; stelios@imada.sdu.dk
  \qquad
  \faGlobe \;
  \href{https://www.steliostsampas.com}{https://www.steliostsampas.com}
  \\\\\
  \faGithub \; stelios-tau/cp-2025
  \qquad\;\;
    \faDiscord \; cp-2025
}
\date{February 17, 2025}

\titlegraphic{\includegraphics[height=0.6cm,keepaspectratio]{../media/sdu-black.eps}}

\usetheme[block=fill]{metropolis}


%\usepackage{pres-common}
\usepackage{textpos}
\usepackage{centernot}

% \newcommand{\Goesv}[3]{\ensuremath{#1 \xRightarrow{~#3~} #2}}
% \newcommand{\goesv}[3]{\ensuremath{#1 \xrightarrow{~#3~} #2}}

% \usepackage{etex}
% \usepackage{semantic}

\usepackage[utf8]{inputenc}
\usepackage[english]{babel}
\usepackage{tikz}
\usepackage{hyperref}

\usetikzlibrary{arrows,shapes,matrix}
\usetikzlibrary{backgrounds}
\usetikzlibrary{positioning}
\usetikzlibrary{automata}
\usetikzlibrary{mindmap}
\usetikzlibrary{shapes.callouts}
\usetikzlibrary{decorations.text}
\usetikzlibrary{tikzmark}
\usetikzlibrary{calc}
\usetikzlibrary{overlay-beamer-styles}
\usetikzlibrary{shapes.geometric}

\tikzset{onslide/.code args={<#1>#2}{%
    \only<#1>{\pgfkeysalso{#2}} % \pgfkeysalso doesn't change the path
  }}

\setbeamercolor{mygray}{bg=Gray!20}

\tikzset{temporal/.code args={<#1>#2#3#4}{%
    \temporal<#1>{\pgfkeysalso{#2}}{\pgfkeysalso{#3}}{\pgfkeysalso{#4}} % \pgfkeysalso doesn't change the path
  }}

\tikzstyle{highlight}=[fill=green!50]
\tikzstyle{hgreen}=[fill=green!20]
\tikzstyle{hred}=[fill=red!50]
\tikzstyle{hblue}=[fill=blue!50]
\tikzstyle{hgray}=[fill=gray!50]

\addtobeamertemplate{frametitle}{}{%
\begin{textblock*}{100mm}(\textwidth-2cm,-0.86cm)
\includegraphics[height=0.6cm,keepaspectratio]{../media/sdu-white.eps}
\end{textblock*}}


%\usepackage{tikz-cd}
% \usepackage{wasysym}
% \usepackage{color}
% \usepackage[matrix,arrow]{xy}
% \xyoption{all}
% \SelectTips{cm}{}
% % \usepackage{cite}
% \usepackage{amsthm}
% \usepackage{amsmath}
% \usepackage{bbold}
% % \usepackage[bbgreekl]{mathbbol}
% \usepackage{amssymb}
% \usepackage{pifont}
% \usepackage{mathtools}
% \usepackage{amsbsy}
% % \usepackage{paralist}
% \usepackage{shadethm}
% % \usepackage{fancyhdr}
% \usepackage{stmaryrd}
% \usepackage{wasysym}
% \usepackage{graphicx}
% \usepackage{tabularx}
% \usepackage{dsfont}
% \usepackage{ulem}




%\bibliography{mainBiblio}

%\includeonlyframes{current}
\begin{document}

\frame{\titlepage}

\def\firstcircle{(0,0) circle (2cm)}
\def\secondcircle{(1.4,1.4) circle (2cm)}
\def\thirdcircle{(0:2.4) circle (2cm)}

\begin{frame}{Outline}
  \tableofcontents
\end{frame}

\section{Recap and introduction}

\begin{frame}[fragile]
  \frametitle{Last week's topics}

  Last week, we touched upon...

  \begin{itemize}
  \item[\faBook]<1-> The fundamental problem of sharing mutable state.
  \item[\faBook]<1-> Atomicity, i.e. the property of a sequence of
    statements appearing \emph{indivisible} w.r.t. other threads.
  \item[\faBook]<1-> How to achieve atomicity and protect code against
    concurrent access by other threads using \emph{intrinsic locks}.
  \item[\faUserInjured]<1-> Several examples of race conditions and how to fix them.
  \end{itemize}
\end{frame}

\begin{frame}[fragile]
  \frametitle{Last week vs this week}

  \large{Last week was more about specific problematic patterns occuring during
    the execution of a \emph{program}.}

  \pause

  \large{This week will be more about \emph{designing} software that can be
    \emph{utilized} safely in a concurrent setting.}

\end{frame}

\begin{frame}[fragile]
  \frametitle{This week's topics}

  This week, we will witness that

  \begin{itemize}
  \item[\faBook]<1-> Concurrency in Java subverts common sense
    on a concurrent setting, due to \emph{visibility}.
  \item[\faBook]<1-> Preparing to \emph{share} an object across threads
    is not as simple as it sounds, due to timing and visibility problems.
  \item[\faBook]<1-> Notions such as \emph{invariants}, \emph{immutability} and
    \emph{encapsulation} play a central role in designing thread-safe classes.
  \end{itemize}

  \begin{itemize}
  \item[\faUserInjured]<2-> We will also demonstrate all of the above in
    examples.
  \end{itemize}
\end{frame}


\section{Visibility}

\begin{frame}[fragile]
  \frametitle{More than just locking}

  \begin{itemize}
  \item[\faBook]<1-> Intrinsic locks ensure that only one thread may enter any
    piece of code that is protected by the same lock.
  \item[\faBook]<1-> It also has a second purpose: it makes updates in one
    thread \emph{visible} to other threads!
    \begin{itemize}
    \item[\faUserInjured]<2-> Without synchronization, this might not happen!
    \end{itemize}
  \end{itemize}

\end{frame}

\begin{frame}[fragile]
  \frametitle{Is this thread-safe?}

\begin{lstlisting}[language = Java , frame = trBL , firstnumber = last ,
escapeinside={(*@}{@*)}]
class KeepsGoing extends Thread {
    boolean keepRunning = true;

    public void run() {
        while (keepRunning) {
            doSomething();
        }
    }
}
\end{lstlisting}

  \uncover<2->{
    \begin{tikzpicture}[overlay, remember picture]
      \node[xshift=10.4cm,yshift=3cm,starburst,starburst points=20,
      align=center,fill=yellow, opacity=1,draw=red, line width=2pt]
      {\textbf{No! \faSkullCrossbones}};
    \end{tikzpicture}}

  \uncover<3->{
    \begin{itemize}
    \item[\faUserInjured] There is no guarantee that changes to
      \texttt{keepRunning} by other threads will be visible in this thread.
    \item[\faBriefcaseMedical] We have to use synchronization!
    \end{itemize}}

\end{frame}

\begin{frame}[fragile]
  \frametitle{Visibility}

  \begin{itemize}
  \item[\faBook] The developer might expect that writes should be
    \emph{visible} to any access that took place \emph{after} (in terms of
    universal time) a write.
  \item[\faUserInjured] However, this is \textbf{not} the case:
    \begin{itemize}
    \item[\faBook] Java does not guarantee when a write in some thread will be
      visible to other threads.
    \item[\faBook] \uncover<2->{Java doesn't even guarantee that writes will be visible
      \textbf{at all}!!}
    \end{itemize}
  \item[\faUserInjured] The culprit here are optimization taking place by Java.
    They are only correctness-preserving in a single-threaded environment.
  \end{itemize}
\end{frame}

\begin{frame}[fragile]
  \frametitle{Visibility and stale data}

  \begin{block}{\faUserInjured\quad Stale data}
    Accessing out-of-date data due to the state of data (i.e. a variable being
    written upon) not being up-to-date across different threads. May lead to
    serious correctness issues.
  \end{block}
  \vspace{0.2cm}
  \begin{block}<2->{\faBriefcaseMedical\quad Solution}
    Always use synchronization to access shared data!
  \end{block}

  \begin{itemize}
  \item[\faBook]<2-> Synchronization causes writes to be visible across threads!
  \end{itemize}
\end{frame}

\begin{frame}[fragile]
  \frametitle{Visibility}

  \begin{center}
    \includegraphics[height=7cm,keepaspectratio]{../media/visibility.png}
  \end{center}

\end{frame}

\begin{frame}[fragile]
  \frametitle{Is this thread-safe (2)}

\begin{lstlisting}[language = Java , frame = trBL , firstnumber = last ,
escapeinside={(*@}{@*)}]
public class MutableInteger {
    private int value;
}
public int get() { return value; }
public void set(int value) { this.value = value; }
\end{lstlisting}

  \uncover<2->{
    \begin{tikzpicture}[overlay, remember picture]
      \node[xshift=10.4cm,yshift=3cm,starburst,starburst points=20,
      align=center,fill=yellow, opacity=1,draw=red, line width=2pt]
      {\textbf{No! \faSkullCrossbones}};
    \end{tikzpicture}}

  \pause
\begin{lstlisting}[language = Java , frame = trBL , firstnumber = last ,
escapeinside={(*@}{@*)}]
public class MutableInteger {
    private int value;
}
public int get() { return value; }
public synchronized void set(int value) { this.value = value; }
\end{lstlisting}

  \uncover<4->{
    \begin{tikzpicture}[overlay, remember picture]
      \node[xshift=10.4cm,yshift=3cm,starburst,starburst points=20,
      align=center,fill=yellow, opacity=1,draw=red, line width=2pt]
      {\textbf{No! \faSkullCrossbones}};
    \end{tikzpicture}}
\end{frame}

\begin{frame}[fragile]
  \frametitle{Is this thread-safe (2)}

\begin{lstlisting}[language = Java , frame = trBL , firstnumber = last ,
escapeinside={(*@}{@*)}]
public class MutableInteger {
    private int value;
}
public synchronized int get() { return value; }
public synchronized void set(int value) { this.value = value; }
\end{lstlisting}

  \uncover<2->{
    \begin{tikzpicture}[overlay, remember picture]
      \node[xshift=10.4cm,yshift=3cm,starburst,starburst points=10,
      align=center,fill=green!50, opacity=1,draw=pink!50, line width=2pt]
      {\textbf{Yes}};
    \end{tikzpicture}}

  \begin{itemize}
  \item[\faBriefcaseMedical]<3-> The moral of the story is, intrinsic locks are
    not just for synchronization, but also for \textbf{visibility}. Use them
    when accessing shared variables, regardless if there are any apparent race
    conditions or not.
  \end{itemize}

\end{frame}

\begin{frame}[fragile]
  \frametitle{Out-of-thin-air safety}

  \begin{itemize}
  \item[\faBook] Staleness is not all-or-nothing
    \begin{itemize}
    \item[\faBook] One value might be fresh while another might be stale.
    \end{itemize}
  \item[\faBook] There is \textbf{almost} one thing that the developer can be sure when
    accessing stale data.
    \begin{itemize}
    \item[\faBook] Any value accessed to a shared memory location was caused by
      a write, and did not appear out of nowhere.
    \item[\faBook] This is known as \emph{out-of-thin-air} safety.
    \end{itemize}
  \item<2->[\faBook] It applies to \textbf{almost} any variable out there.
  \item<3->[\faBook] Except \texttt{double}'s and \texttt{long}'s.
  \end{itemize}

  \uncover<4->{By the way, remember DataRace.java?}

\end{frame}

\begin{frame}[fragile]
  \frametitle{Volatility}

  \begin{block}{\texttt{Volatile} variables}
    The \texttt{volatile} modifier ensures that an attribute's value is always the same
    when read from any thread.
  \end{block}

  \begin{itemize}
  \item[\faBook]<1-> One sneaky way to ensure visibility is to use
    \texttt{volatile} variables.
  \item[\faBook]<1-> They achieve synchronization, as Java ensures that a read
    to a volatile variable will \emph{always} return the most recent value.
  \item[\faBook]<1-> They are neat and lightweight.
  \item[\faBook]<1-> Like locks, they also ensure that all variables prior to
    reading/writing a volatile varible are \emph{visible} across threads.
  \end{itemize}

\end{frame}

\begin{frame}[fragile]
  \frametitle{Volatility}

  \begin{block}{\faBook \quad Remember!}
    Locking can guarantee both visibility and atomicity; volatile variables can
    only guarantee visibility.
  \end{block}

  You should use volatile variables only when all the following criteria are met:

  \begin{itemize}
  \item[\faBook]<1-> Writes to the variable do not depend on its current value,
    or you can ensure that only a single thread ever updates the value;
  \item[\faBook]<1-> The variable does not participate in invariants with other
    state variables, and
  \item[\faBook]<1-> Locking is not required for any other reason while the variable is being
    accessed.
  \end{itemize}

\end{frame}

\section{Publishing and sharing}

\begin{frame}[fragile]
  \frametitle{Publishing an object}

  \begin{itemize}
  \item[\faBook]<1-> You publish an object when it becomes available outside
    each scope, e.g. by passing its reference around.
  \item[\faBook]<1-> It is easy to inadvertently publish an object, so it is
    important to understand when publishing happens.
    \begin{itemize}
    \item[\faBook]<2-> We say that such objects have \emph{escaped}.
    \end{itemize}
  \item[\faUserInjured]<1-> There are also various pitfalls associated with
    publishing in a concurrent setting.
  \end{itemize}
\end{frame}

\begin{frame}[fragile]
  \frametitle{Publishing an object}

  \begin{itemize}
  \item[\faBook]<1-> Publishing an object can be as simple as using a public variable.
  \end{itemize}
\begin{lstlisting}[language = Java , frame = trBL , firstnumber = last ,
escapeinside={(*@}{@*)}]
public static Set<Secret> knownSecrets;
public void initialize() {
    knownSecrets = new HashSet<Secret>();
}
\end{lstlisting}
  \pause
  \begin{itemize}
  \item[\faBook]<1-> Or returning an object.
  \end{itemize}
\begin{lstlisting}[language = Java , frame = trBL , firstnumber = last ,
escapeinside={(*@}{@*)}]
class UnsafeStates {
    private String[] states = 
        new String[] {"AK", "AL" ...};
    public String[] getStates() { return states; }
}
\end{lstlisting}
  \uncover<3->{
    \begin{tikzpicture}[overlay, remember picture]
      \node[xshift=10.4cm,yshift=3cm,starburst,starburst points=40,
      align=center,fill=yellow, opacity=1,draw=red, line width=2pt]
      {\textbf{A \texttt{private} var gone public \faSkullCrossbones!}};
    \end{tikzpicture}}
\end{frame}

\begin{frame}[fragile]
  \frametitle{Publishing an object}

  \begin{itemize}
  \item[\faBook]<1-> All the way to rather obscure ways.
  \end{itemize}
\begin{lstlisting}[language = Java , frame = trBL , firstnumber = last ,
escapeinside={(*@}{@*)}]
public class ThisEscape {
    public ThisEscape(EventSource source) {
        source.registerListener(
            new EventListener() {
                public void onEvent(Event e) {
                    doSomething(e);
                }
            });
    }
}
\end{lstlisting}
  \uncover<2->{
    \begin{tikzpicture}[overlay, remember picture]
      \node[xshift=10.4cm,yshift=2cm,starburst,starburst points=30,
      align=center,fill=yellow, opacity=1,draw=red, line width=2pt]
      {\textbf{\texttt{this} escape! \faSkullCrossbones}};
    \end{tikzpicture}}
\end{frame}

\begin{frame}[fragile]
  \frametitle{\texttt{this} escape}

\begin{lstlisting}[language = Java , frame = trBL , firstnumber = last ,
escapeinside={(*@}{@*)}]
public class ThisEscape
    public ThisEscape(EventSource source)
        source.registerListener(
            new EventListener()
                public void onEvent(Event e)
                    doSomething(e);)
\end{lstlisting}

  \begin{itemize}
  \item[\faBook]<1-> After the \texttt{new} keyword, an anonymous class is being
    created, implementing \texttt{EventListener}.
  \item[\faBook]<2-> This class is shared to \texttt{source.registerListener}.
  \item[\faBook]<3-> Inner classes carry an implicit reference to \texttt{this}.
  \item[\faBook]<4-> Hence, \texttt{this} is shared... \pause in its
    \textbf{constructor} \faSkullCrossbones!
  \end{itemize}
\end{frame}

\begin{frame}[fragile]
  \frametitle{\texttt{this} escape (in constructor)}

\begin{lstlisting}[language = Java , frame = trBL , firstnumber = last ,
escapeinside={(*@}{@*)}]
public class ThisEscape
    public ThisEscape(EventSource source)
        source.registerListener(
            new EventListener()
                public void onEvent(Event e)
                    doSomething(e);)
\end{lstlisting}

  \begin{block}{\texttt{\faBook} Important!}
    Do not allow the this reference to escape during construction. Another
    thread might get a hold of an object before it is properly initialized.
  \end{block}

  \begin{itemize}
  \item[\faBriefcaseMedical] A solution could(!) be to initalize the object in
    static method.
  \end{itemize}
\end{frame}

\begin{frame}[fragile]
  \frametitle{How not to share}

  \begin{block}{Pro tip (ad-hoc confinement)}
    If you can \emph{ensure} that an object always stays within a thread, no
    synchronization is needed.
  \end{block}
  \vspace{0.2cm}
  \begin{block}{Stack confinement}
    Making sures objects can only be reached through local variables. As local
    variables are intrinsically tied to the execution of a thread, they cannot escape.
  \end{block}
  \vspace{0.2cm}
  \begin{block}{ThreadLocal confinement}
    Programmatic way to ensure that each thread keeps its own version of the object.
  \end{block}
\end{frame}

\begin{frame}[fragile]
  \frametitle{Immutability}
\end{frame}

\begin{frame}[fragile]
  \frametitle{Techniques for safe sharing}
\end{frame}

\section{Designing thread-safe classes}

\begin{frame}{}
  \centering \huge
  Thank you!
\end{frame}

\end{document}

%%% Local Variables:
%%% mode: latex
%%% TeX-engine: xetex
%%% TeX-master: t
%%% End:
