
\PassOptionsToPackage{colorlinks,linkcolor={blue},citecolor={blue},urlcolor={blue},breaklinks=true,final}{hyperref}
\PassOptionsToPackage{dvipsnames}{xcolor}
\documentclass[xcolor={dvipsnames,svgnames},aspectratio=169]{beamer}

\usepackage{fontawesome5}
\usepackage{booktabs} % For better table formatting
\usepackage{listings}
\usepackage{tabularx}


\lstset{
  tabsize = 4, %% set tab space width
  showstringspaces = false, %% prevent space marking in strings, string is defined as the text that is generally printed directly to the console
  numbers = left, %% display line numbers on the left
  commentstyle = \color{purple!60}, %% set comment color
  keywordstyle = \color{blue}, %% set keyword color
  stringstyle = \color{red}, %% set string color
  rulecolor = \color{black}, %% set frame color to avoid being affected by text color
  basicstyle = \small \ttfamily , %% set listing font and size
  breaklines = true, %% enable line breaking
  numberstyle = \tiny,
}

\title{Concurrent Programming}
\subtitle{Week 18 (Lecture 7) : \textbf{Liveness hazards and performance}}
\author{Stelios Tsampas}
\institute{
  \faEnvelope \; stelios@imada.sdu.dk
  \qquad
  \faGlobe \;
  \href{https://www.steliostsampas.com}{https://www.steliostsampas.com}
  \\\\\
  \faGithub \; stelios-tau/cp-2025
  \qquad\;\;
    \faDiscord \; cp-2025
}
\date{\today}

\titlegraphic{\includegraphics[height=0.6cm,keepaspectratio]{../media/sdu-black.eps}}

\usetheme[block=fill]{metropolis}


%\usepackage{pres-common}
\usepackage{textpos}
\usepackage{centernot}

% \newcommand{\Goesv}[3]{\ensuremath{#1 \xRightarrow{~#3~} #2}}
% \newcommand{\goesv}[3]{\ensuremath{#1 \xrightarrow{~#3~} #2}}

% \usepackage{etex}
% \usepackage{semantic}

\usepackage[utf8]{inputenc}
\usepackage[english]{babel}
\usepackage{tikz}
\usepackage{hyperref}

\usetikzlibrary{arrows,shapes,matrix}
\usetikzlibrary{backgrounds}
\usetikzlibrary{positioning}
\usetikzlibrary{automata}
\usetikzlibrary{mindmap}
\usetikzlibrary{shapes.callouts}
\usetikzlibrary{decorations.text}
\usetikzlibrary{tikzmark}
\usetikzlibrary{calc}
\usetikzlibrary{overlay-beamer-styles}
\usetikzlibrary{shapes.geometric}
\usepackage{pgfplots}


\tikzset{onslide/.code args={<#1>#2}{%
    \only<#1>{\pgfkeysalso{#2}} % \pgfkeysalso doesn't change the path
  }}

\setbeamercolor{mygray}{bg=Gray!20}

\tikzset{temporal/.code args={<#1>#2#3#4}{%
    \temporal<#1>{\pgfkeysalso{#2}}{\pgfkeysalso{#3}}{\pgfkeysalso{#4}} % \pgfkeysalso doesn't change the path
  }}

\tikzstyle{highlight}=[fill=green!50]
\tikzstyle{hgreen}=[fill=green!20]
\tikzstyle{hred}=[fill=red!50]
\tikzstyle{hblue}=[fill=blue!50]
\tikzstyle{hgray}=[fill=gray!50]

\addtobeamertemplate{frametitle}{}{%
\begin{textblock*}{100mm}(\textwidth-2cm,-0.86cm)
\includegraphics[height=0.6cm,keepaspectratio]{../media/sdu-white.eps}
\end{textblock*}}


%\usepackage{tikz-cd}
% \usepackage{wasysym}
% \usepackage{color}
% \usepackage[matrix,arrow]{xy}
% \xyoption{all}
% \SelectTips{cm}{}
% % \usepackage{cite}
% \usepackage{amsthm}
% \usepackage{amsmath}
% \usepackage{bbold}
% % \usepackage[bbgreekl]{mathbbol}
% \usepackage{amssymb}
% \usepackage{pifont}
% \usepackage{mathtools}
% \usepackage{amsbsy}
% % \usepackage{paralist}
% \usepackage{shadethm}
% % \usepackage{fancyhdr}
% \usepackage{stmaryrd}
% \usepackage{wasysym}
% \usepackage{graphicx}
% \usepackage{tabularx}
% \usepackage{dsfont}
% \usepackage{ulem}




%\bibliography{mainBiblio}

%\includeonlyframes{current}
\begin{document}

\frame{\titlepage}

\def\firstcircle{(0,0) circle (2cm)}
\def\secondcircle{(1.4,1.4) circle (2cm)}
\def\thirdcircle{(0:2.4) circle (2cm)}

\begin{frame}{Outline}
  \tableofcontents
\end{frame}

\section{Recap}

\begin{frame}[fragile]
  \frametitle{Previously on CP}

  Last lecture, we looked at...

  \begin{itemize}
  \item[\faBook]<1-> The fundamental producer-consumer pattern, in which the
    creation of tasks is separated from their undertaking.
    \begin{itemize}
    \item[\faBook]<1-> \emph{Producer} threads create tasks, \emph{consumers}
      complete them.
    \end{itemize}
  \item[\faBook]<1-> Low-level thread coordination primitives
    \begin{itemize}
    \item[\faBook]<1-> Namely \texttt{wait()}, \texttt{notify()} and
      \texttt{notifyAll()}.
    \end{itemize}
  \item[\faBook]<1-> The \texttt{BlockingQueue} interface as an efficient way to
    manage tasks.
  \item[\faBook]<1-> The \texttt{Executor} as an efficient, more advanced way to
    manage tasks.
  \end{itemize}

  \vspace{0.4cm}

  \begin{block}<2->{\faLightbulb \quad Key takeaway}
    The producer-consumer is a ubiquitous pattern arises naturally in many concurrency
    scenarios (it is not the only one). The developer has many tools to manage
    tasks in an efficient and reliable manner.
  \end{block}

\end{frame}

\begin{frame}[fragile]
  \frametitle{This week's topics}

  This week, we will look at...

  \begin{itemize}
  \item[\faBook]<1-> Liveness issues!
  \item[\faBook]<1-> Performance and scalability considerations!
  \item[\faBook]<1-> Streams in parallel!
  \item[\faBook]<1-> Some information on the exam!
  \end{itemize}
\end{frame}

\begin{frame}[fragile]
  \frametitle{Hazards in concurrency}

  In concurrent programming, there are three main classes of hazards:

  \begin{itemize}
  \item[\faUserInjured]<1-> (Thread) safety issues (e.g. race conditions,
    publishing etc.).
  \item[\faUserInjured]<2-> \textbf{Liveness issues}.
  \item[\faUserInjured]<3-> \textbf{Performance}.
  \end{itemize}

\end{frame}

\section{Liveness Hazards}



\begin{frame}[fragile]
  \frametitle{Liveness in Concurrency}

  \begin{block}<1->{\faSearch \quad Liveness}
    Something good eventually happens.
  \end{block}

  In concurrent programs, \textbf{liveness hazards} occur when:

  \begin{itemize}
  \item[\faUserInjured]<1-> Threads \textbf{don't make progress} toward their
    goals.
  \item[\faUserInjured]<1-> Systems appear stuck or unresponsive.
  \end{itemize}
\end{frame}

\begin{frame}[fragile]
  \frametitle{Liveness in Concurrency}

  \begin{block}<1->{\faSearch \quad Liveness}
    \emph{Something good eventually happens.}
  \end{block}

  In concurrent programs, \textbf{liveness hazards} occur when:

  \begin{itemize}
  \item[\faUserInjured]<1-> Threads \textbf{don't make progress} toward their
    goals.
  \item[\faUserInjured]<1-> Systems appear stuck or unresponsive.
  \end{itemize}

  The three classic problems are \textbf{Deadlock}, \textbf{starvation} and
  \textbf{livelock}.

  \vspace{0.2cm}

  \begin{block}<2->{\faExclamationTriangle \quad Warning}
    Liveness issues are elusive, do not manifest easily, so extra care is
    needed.
  \end{block}

\end{frame}

\begin{frame}[fragile]
  \frametitle{\faSkull \quad Deadlock}

  \begin{block}<1->{\faSearch \quad What is a deadlock?}
    The condition during execution where two or more threads are blocked
    forever, waiting for each other.
  \end{block}

  \begin{block}<1->{\faSearch \quad Real-world example}
    Two cars are stuck on a one-way bridge, blocking each other.
  \end{block}


\end{frame}

\begin{frame}[fragile]
  \frametitle{Starvation}

  \begin{block}<1->{\faSearch \quad Real-world example}
    Waiting forever in the emergency room in a hospital because more serious
    cases keep coming up. \uncover<2->{We've all been there.}
  \end{block}

\end{frame}

\begin{frame}[fragile]
  \frametitle{Livelock}

  \begin{block}<1->{\faSearch \quad Real-world example}
    Two friends keep calling each other because the line is busy.
    \uncover<2->{We've all been there.}
  \end{block}

\end{frame}

\section{Performance and Scalability}

\section{(Parallel) Streams}

\section{The Exam}

\begin{frame}{}
  \centering \huge
  Thank you!
\end{frame}

\end{document}

%%% Local Variables:
%%% mode: latex
%%% TeX-engine: xetex
%%% TeX-master: t
%%% End:
