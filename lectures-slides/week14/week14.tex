
\PassOptionsToPackage{colorlinks,linkcolor={blue},citecolor={blue},urlcolor={blue},breaklinks=true,final}{hyperref}
\PassOptionsToPackage{dvipsnames}{xcolor}
\documentclass[xcolor={dvipsnames,svgnames},aspectratio=169]{beamer}

\usepackage{fontawesome5}
\usepackage{booktabs} % For better table formatting
\usepackage{listings}
\usepackage{tabularx}


\lstset{
  tabsize = 4, %% set tab space width
  showstringspaces = false, %% prevent space marking in strings, string is defined as the text that is generally printed directly to the console
  numbers = left, %% display line numbers on the left
  commentstyle = \color{purple!60}, %% set comment color
  keywordstyle = \color{blue}, %% set keyword color
  stringstyle = \color{red}, %% set string color
  rulecolor = \color{black}, %% set frame color to avoid being affected by text color
  basicstyle = \small \ttfamily , %% set listing font and size
  breaklines = true, %% enable line breaking
  numberstyle = \tiny,
}

\title{Concurrent Programming}
\subtitle{Week 14 (Lecture 6) : \textbf{Queues, tasks and executors}}
\author{Stelios Tsampas}
\institute{
  \faEnvelope \; stelios@imada.sdu.dk
  \qquad
  \faGlobe \;
  \href{https://www.steliostsampas.com}{https://www.steliostsampas.com}
  \\\\\
  \faGithub \; stelios-tau/cp-2025
  \qquad\;\;
    \faDiscord \; cp-2025
}
\date{\today}

\titlegraphic{\includegraphics[height=0.6cm,keepaspectratio]{../media/sdu-black.eps}}

\usetheme[block=fill]{metropolis}


%\usepackage{pres-common}
\usepackage{textpos}
\usepackage{centernot}

% \newcommand{\Goesv}[3]{\ensuremath{#1 \xRightarrow{~#3~} #2}}
% \newcommand{\goesv}[3]{\ensuremath{#1 \xrightarrow{~#3~} #2}}

% \usepackage{etex}
% \usepackage{semantic}

\usepackage[utf8]{inputenc}
\usepackage[english]{babel}
\usepackage{tikz}
\usepackage{hyperref}

\usetikzlibrary{arrows,shapes,matrix}
\usetikzlibrary{backgrounds}
\usetikzlibrary{positioning}
\usetikzlibrary{automata}
\usetikzlibrary{mindmap}
\usetikzlibrary{shapes.callouts}
\usetikzlibrary{decorations.text}
\usetikzlibrary{tikzmark}
\usetikzlibrary{calc}
\usetikzlibrary{overlay-beamer-styles}
\usetikzlibrary{shapes.geometric}
\usepackage{pgfplots}


\tikzset{onslide/.code args={<#1>#2}{%
    \only<#1>{\pgfkeysalso{#2}} % \pgfkeysalso doesn't change the path
  }}

\setbeamercolor{mygray}{bg=Gray!20}

\tikzset{temporal/.code args={<#1>#2#3#4}{%
    \temporal<#1>{\pgfkeysalso{#2}}{\pgfkeysalso{#3}}{\pgfkeysalso{#4}} % \pgfkeysalso doesn't change the path
  }}

\tikzstyle{highlight}=[fill=green!50]
\tikzstyle{hgreen}=[fill=green!20]
\tikzstyle{hred}=[fill=red!50]
\tikzstyle{hblue}=[fill=blue!50]
\tikzstyle{hgray}=[fill=gray!50]

\addtobeamertemplate{frametitle}{}{%
\begin{textblock*}{100mm}(\textwidth-2cm,-0.86cm)
\includegraphics[height=0.6cm,keepaspectratio]{../media/sdu-white.eps}
\end{textblock*}}


%\usepackage{tikz-cd}
% \usepackage{wasysym}
% \usepackage{color}
% \usepackage[matrix,arrow]{xy}
% \xyoption{all}
% \SelectTips{cm}{}
% % \usepackage{cite}
% \usepackage{amsthm}
% \usepackage{amsmath}
% \usepackage{bbold}
% % \usepackage[bbgreekl]{mathbbol}
% \usepackage{amssymb}
% \usepackage{pifont}
% \usepackage{mathtools}
% \usepackage{amsbsy}
% % \usepackage{paralist}
% \usepackage{shadethm}
% % \usepackage{fancyhdr}
% \usepackage{stmaryrd}
% \usepackage{wasysym}
% \usepackage{graphicx}
% \usepackage{tabularx}
% \usepackage{dsfont}
% \usepackage{ulem}




%\bibliography{mainBiblio}

%\includeonlyframes{current}
\begin{document}

\frame{\titlepage}

\def\firstcircle{(0,0) circle (2cm)}
\def\secondcircle{(1.4,1.4) circle (2cm)}
\def\thirdcircle{(0:2.4) circle (2cm)}

\begin{frame}{Outline}
  \tableofcontents
\end{frame}

\section{Recap}

\begin{frame}[fragile]
  \frametitle{Last week's topics}

  Last lecture, we touched upon...

  \begin{itemize}
  \item[\faBook]<1-> Spinlock, or locking via busy-waiting, and noted how slow
    spinlocks are on the Java level.
  \item[\faBook]<1-> Using \texttt{CountDownLatch}'s to elegantly manage and
    bookkeep thread termination.
    \begin{itemize}
    \item[\faBook]<1-> In the process, we briefly broached the subject of task
      delegation.
    \end{itemize}
  \item[\faBook]<1-> Using Java's synchronized collections to simplify
    concurrency.
  \item[\faBook]<1-> Discussed various matters of performance and efficiency.
  \end{itemize}

  \vspace{0.4cm}

  \begin{block}<2->{\faLightbulb \quad Key takeaway}
    To maintain optimal performance, one should i) \textbf{minimize blocking code}
    (achieved often by proper \emph{task delegation}) and ii) use the most
    \textbf{efficient data structures} for concurrency.
  \end{block}
\end{frame}

\begin{frame}[fragile]
  \frametitle{\texttt{synchronizedMap()} vs \texttt{ConcurrentHashMap}}


\end{frame}

\begin{frame}[fragile]
  \frametitle{This week's topics}

  This week, we will look at...

  \begin{itemize}
  \item[\faBook]<1-> The \emph{producer-consumer} pattern.
  \item[\faBook]<1-> \texttt{Notify}-ing and \texttt{Wait}-ing.
  \item[\faBook]<1-> Task management with \texttt{BlockingQueue}'s.
  \item[\faBook]<1-> Task management with \texttt{Executor}'s.
  \end{itemize}


\end{frame}



\begin{frame}[fragile]
  \frametitle{Busy-wait}

\begin{lstlisting}[language = Java , frame = trBL , firstnumber = last ,
escapeinside={(*@}{@*)}]
while ((something) == 0) {
        /* do nothing - just keep checking over and over */
    }

doSomething();
\end{lstlisting}

  \vspace{0.2cm}
  \begin{block}<1->{\faLightbulb \quad Key idea}
    Wait -- and keep checking -- until a condition is met.
  \end{block}
\end{frame}



\begin{frame}[fragile]
  \frametitle{Code Listings}

  \begin{itemize}
  \item[\faCode]<1-> LockTestHarness/Spinlock.java: Spinlock implementation
  \item[\faCode]<1-> LockTestHarness/LockTestHarness.java: Spinlock vs
    \texttt{synchronized} (single-run)
  \item[\faCode]<1-> LockTestHarness/LockTestHarness2.java: Spinlock vs
    \texttt{synchronized} (benchmark)
  \item[\faCode]<1-> LatchTestHarness/LatchTestHarness.java: Latch (local
    updates) vs
    \texttt{synchronized} (single)
  \item[\faCode]<1-> LatchTestHarness/LatchTestHarness2.java: Latch (local
    updates) vs
    \texttt{synchronized} (benchmark)
  \item[\faCode]<1-> LatchTestHarness/LatchTestHarness2Long.java: Proper
    benchmarking for latching and local updates using \texttt{long}'s.
  \item[\faCode]<1-> LatchTestHarness/LatchTestHarness2LongMap.java: Latch
    (local updates) vs \texttt{ConcurrentHashMap}.
  \end{itemize}

\end{frame}

\begin{frame}{}
  \centering \huge
  Thank you!
\end{frame}

\end{document}

%%% Local Variables:
%%% mode: latex
%%% TeX-engine: xetex
%%% TeX-master: t
%%% End:
