
\PassOptionsToPackage{colorlinks,linkcolor={blue},citecolor={blue},urlcolor={blue},breaklinks=true,final}{hyperref}
\PassOptionsToPackage{dvipsnames}{xcolor}
\documentclass[xcolor={dvipsnames,svgnames},aspectratio=169]{beamer}

\usepackage{fontawesome5}
\usepackage{booktabs} % For better table formatting
\usepackage{listings}

\lstset{
  tabsize = 4, %% set tab space width
  showstringspaces = false, %% prevent space marking in strings, string is defined as the text that is generally printed directly to the console
  numbers = left, %% display line numbers on the left
  commentstyle = \color{purple!60}, %% set comment color
  keywordstyle = \color{blue}, %% set keyword color
  stringstyle = \color{red}, %% set string color
  rulecolor = \color{black}, %% set frame color to avoid being affected by text color
  basicstyle = \small \ttfamily , %% set listing font and size
  breaklines = true, %% enable line breaking
  numberstyle = \tiny,
}

\title{Concurrent Programming}
\subtitle{Week 13 (Lecture 5)}
\author{Stelios Tsampas}
\institute{
  \faEnvelope \; stelios@imada.sdu.dk
  \qquad
  \faGlobe \;
  \href{https://www.steliostsampas.com}{https://www.steliostsampas.com}
  \\\\\
  \faGithub \; stelios-tau/cp-2025
  \qquad\;\;
    \faDiscord \; cp-2025
}
\date{\today}

\titlegraphic{\includegraphics[height=0.6cm,keepaspectratio]{../media/sdu-black.eps}}

\usetheme[block=fill]{metropolis}


%\usepackage{pres-common}
\usepackage{textpos}
\usepackage{centernot}

% \newcommand{\Goesv}[3]{\ensuremath{#1 \xRightarrow{~#3~} #2}}
% \newcommand{\goesv}[3]{\ensuremath{#1 \xrightarrow{~#3~} #2}}

% \usepackage{etex}
% \usepackage{semantic}

\usepackage[utf8]{inputenc}
\usepackage[english]{babel}
\usepackage{tikz}
\usepackage{hyperref}

\usetikzlibrary{arrows,shapes,matrix}
\usetikzlibrary{backgrounds}
\usetikzlibrary{positioning}
\usetikzlibrary{automata}
\usetikzlibrary{mindmap}
\usetikzlibrary{shapes.callouts}
\usetikzlibrary{decorations.text}
\usetikzlibrary{tikzmark}
\usetikzlibrary{calc}
\usetikzlibrary{overlay-beamer-styles}
\usetikzlibrary{shapes.geometric}

\tikzset{onslide/.code args={<#1>#2}{%
    \only<#1>{\pgfkeysalso{#2}} % \pgfkeysalso doesn't change the path
  }}

\setbeamercolor{mygray}{bg=Gray!20}

\tikzset{temporal/.code args={<#1>#2#3#4}{%
    \temporal<#1>{\pgfkeysalso{#2}}{\pgfkeysalso{#3}}{\pgfkeysalso{#4}} % \pgfkeysalso doesn't change the path
  }}

\tikzstyle{highlight}=[fill=green!50]
\tikzstyle{hgreen}=[fill=green!20]
\tikzstyle{hred}=[fill=red!50]
\tikzstyle{hblue}=[fill=blue!50]
\tikzstyle{hgray}=[fill=gray!50]

\addtobeamertemplate{frametitle}{}{%
\begin{textblock*}{100mm}(\textwidth-2cm,-0.86cm)
\includegraphics[height=0.6cm,keepaspectratio]{../media/sdu-white.eps}
\end{textblock*}}


%\usepackage{tikz-cd}
% \usepackage{wasysym}
% \usepackage{color}
% \usepackage[matrix,arrow]{xy}
% \xyoption{all}
% \SelectTips{cm}{}
% % \usepackage{cite}
% \usepackage{amsthm}
% \usepackage{amsmath}
% \usepackage{bbold}
% % \usepackage[bbgreekl]{mathbbol}
% \usepackage{amssymb}
% \usepackage{pifont}
% \usepackage{mathtools}
% \usepackage{amsbsy}
% % \usepackage{paralist}
% \usepackage{shadethm}
% % \usepackage{fancyhdr}
% \usepackage{stmaryrd}
% \usepackage{wasysym}
% \usepackage{graphicx}
% \usepackage{tabularx}
% \usepackage{dsfont}
% \usepackage{ulem}




%\bibliography{mainBiblio}

%\includeonlyframes{current}
\begin{document}

\frame{\titlepage}

\def\firstcircle{(0,0) circle (2cm)}
\def\secondcircle{(1.4,1.4) circle (2cm)}
\def\thirdcircle{(0:2.4) circle (2cm)}

\begin{frame}{Outline}
  \tableofcontents
\end{frame}

\section{Recap}

\begin{frame}[fragile]
  \frametitle{Last week's topics}

  Last lecture, we touched upon...

  \begin{itemize}
  \item[\faBook]<1-> How to design software that can be
    \emph{utilized} safely in a concurrent setting.
  \item[\faBook]<1-> \emph{Visibility} in all its weirdness.
  \item[\faBook]<1-> The tricky topic of object \emph{sharing}.
  \item[\faBook]<1-> The role of \emph{invariants}, \emph{immutability} and
    \emph{encapsulation}.
  \end{itemize}
\end{frame}

\begin{frame}[fragile]
  \frametitle{Some lessons so far}

  \begin{block}{\faBook\quad It's all about the mutable state.}
    All concurrency issues boil down to coordinating access to mutable
    state. The less mutable state, the easier it is to ensure thread safety.
  \end{block}

  \vspace{0.2cm}

  \begin{block}<2->{\faBook\quad Immutable objects are thread-safe.}
    Immutable objects simplify concurrent programming tremendously.
    They are simpler and safer, and can be shared freely without locking
    or defensive copying.
  \end{block}

  \vspace{0.2cm}

  \begin{block}<3->{\faBook\quad Encapsulation simplifies concurrency.}
    Organizes mutable states, hence organizes the invariants and makes them more
    manageable.
  \end{block}
\end{frame}

\begin{frame}[fragile]
  \frametitle{Some lessons so far (cont.)}

  \begin{itemize}
  \item[\faBook]<1-> If a field is not mutable, make it \textbf{final}.
  \item[\faBook]<1-> Guard each mutable variable with a \textbf{lock}.
  \item[\faBook]<1-> Guard all variables in an \textbf{invariant} with the
    \textbf{same} lock.
  \item[\faBook]<1-> Hold locks for the duration of compound actions.
  \item[\faBook]<1-> Don’t rely on visibility, obscure knowledge of the memory
    model or generally clever reasoning instead of synchronizing.
  \item[\faBook]<1-> Make thread safety a \textbf{central part} of your design
    process.
  \end{itemize}
\end{frame}

\begin{frame}[fragile]
  \frametitle{This week's topics}

  This week, we will witness that

  \begin{itemize}
  \item[\faBook]<1-> Spinlocks!
  \item[\faBook]<1-> Latches!
  \item[\faBook]<1-> Concurrent Collections!
  \end{itemize}

  \begin{itemize}
  \item[\faUserInjured]<2-> We will also demonstrate all of the above in
    examples.
  \end{itemize}
\end{frame}

\section{Spinlocks}

\begin{frame}[fragile]
  \frametitle{Why spin?}

  \begin{columns}
    \begin{column}{0.5\textwidth}
      \begin{itemize}
      \item<1->[\faBook] It is a pattern (busy-wait) that a beginner coder could
        come up with.
      \item<2->[\faBook] Often not the best choice\dots
      \item<3->[\faBook] But situationally it makes sense!
      \end{itemize}

      \vspace{1cm}

      \begin{itemize}
      \item<4->[\faSearch] Basically the Garen of concurrency.
      \end{itemize}
    \end{column}
    \begin{column}{0.5\textwidth}  %%<--- here
      \uncover<4->{
      \begin{center}
        \includegraphics[height=7cm,keepaspectratio]{../media/garen.jpg}
      \end{center}}
    \end{column}
  \end{columns}
\end{frame}

\begin{frame}[fragile]
  \frametitle{What are spinlocks}

  \begin{block}{\faLock\quad Spinlock}
    A type of lock where a thread repeatedly checks if the lock is available.
  \end{block}

  \begin{itemize}
  \item[\faBook]<1-> Busy-waits in a loop ("spins") instead of sleeping.
  \item[\faBook]<1-> Example of a non-blocking synchronization primitive.
  \item[\faBook]<1-> Used when:
    \begin{itemize}
    \item[\faCheck] Critical section is very short.
    \item[\faCheck] Threads are expected to wait only briefly.
    \end{itemize}
  \end{itemize}

  \begin{block}<2->{\faLightbulb \quad Key idea}
    Avoid the overhead of OS thread context switches.
  \end{block}
\end{frame}

\begin{frame}[fragile]
  \frametitle{Busy-wait}

\begin{lstlisting}[language = Java , frame = trBL , firstnumber = last ,
escapeinside={(*@}{@*)}]
while ((something) == 0) {
        /* do nothing - just keep checking over and over */
    }

doSomething();
\end{lstlisting}

  \begin{block}<1->{\faLightbulb \quad Key idea}
    Wait -- and keep checking -- until a condition is met.
  \end{block}
\end{frame}

\begin{frame}[fragile]
  \frametitle{Spinlock vs Synchronized (Blocking)}
  \begin{table}[]
    \centering
    \begin{tabular}{l|l|l}
        \toprule
      \textbf{}
      & \textbf{Spinlock}
      & \textbf{Synchronized / Blocking Lock} \\
        \midrule
      \textbf{Behavior}
      & Active waiting (spins)
      & Passive waiting (thread blocks)\\
        \midrule
      \textbf{CPU Usage}
      & High during contention
      & Low (thread sleeps)\\
        \midrule
      \textbf{Context switch}
      & Avoided (keeps spinning)
      & Incurred (expensive) \\
        \midrule
      \textbf{Best for}
      & Low-contention, short tasks
      & Longer critical sections \\
        \bottomrule
    \end{tabular}
  \end{table}

  \begin{block}<2->{\faExclamationTriangle \quad Important}
    Spinlocks waste CPU if contention is high!
  \end{block}
\end{frame}

\section{Latches}

\section{Concurrent Collections}

\end{document}

%%% Local Variables:
%%% mode: latex
%%% TeX-engine: xetex
%%% TeX-master: t
%%% End:
